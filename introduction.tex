\section{Introduction}\label{intro}
% %%%%%%%%%%%%%%%%%%%%%
Internet topology refers to the study of the various types of
connectivity structures and representations between directly connected nodes on
the Internet architecture~\cite{Calvert97}. This representation aims at
obtaining models that represent  Internet with the greatest possible accuracy in order
to test new communications protocols, algorithms, QoS policies, traffic
engineering, etc.

The Internet topology can be seen at several abstraction levels i.e., IP
interface, router, subnetwork, PoP, and Autonomous System (AS) levels. All these
models have been widely studied in the past~\cite{DONNET13} . However, the
current state of the art of Internet deployments involves a great number of
technologies impacting the Internet Topology.  And those technologies deserve a
deep study in order to include them in Internet topology models.  For instance, 
\dfn{Multiprotocol Label Switching} (MPLS)~\cite{rfc3031} has been recently the
focus of several studies~\cite{SOM11,DONNET13,Vanaubel15}.  It has been
demonstrated that MPLS is a mature technology widely deployed for (mainly) load
balancing reasons or traffic engineering purposes.  A few studies have
partially questioned its impact on Internet topology~\cite{BRICE07,Flach2012}.

Although, the MPLS deployment impact on the Internet architecture has not been
studied yet. The importance to study new architectural details and topological
features related with MPLS usage and its impact would help to know the way in
which the Internet Service Providers (ISPs) use their networks or apply their
policies for traffic engineering as well as to better understand the today's
Internet architecture more accurately. Our work provides a study around the
impact of MPLS deployments over Internet Topology. Principally, we focus on the
properties and features that MPLS modifies on traditional networks maps such as
router level topology. For our purpose we mainly based on  $k$-core
decomposition method~\cite{batagelj2002}. In this way, it has been shown
previously that the $k$-core decomposition is a relevant tool to describe
Internet Topology, by being capable of identifying networks sources by means of
the visualization~\cite{Alvarez06k}, may be used to validate
models~\cite{Serrano06}, and discover exploration biases on the Internet
\cite{Alvarez08k}.

In this way, this paper adds complementary and enriched information around MPLS
usage on Internet. First, we present a quantification of the biases involved on
implicit MPLS tunnel detection. Additionally, we provided some properties and
architectural details related with MPLS deployment on router topology. We found
that routers with lower degree are frequently connected with MPLS capable
routers and routers with high degree are usually connected to common MPLS
networks. Additionally, we describe the behaviour of MPLS networks within some
ASes with most MPLS usage. We found that given an AS, the MPLS networks tend to
form few and well defined clouds. However, this observation could change on
regarding with the type of MPLS tunnel that prevails.

%\ed{BD: describe here quickly our main findings}

The remainder of this paper is organized as follows: Sec.~\ref{related} provides
the state of the art and the background related to MPLS tunnels discovery. In
particular, it describes how MPLS tunnels can be revealed through active
measurements;  Sec.~\ref{dataset} explains how we collected data for this work;
Sec.~\ref{validation} presents our results related to \textit{mpls signatures}
accuracy; Sec.~\ref{cluster} presents the main contributions of this paper with
a detailed study around the behaviour of MPLS networks on the Internet Topology
and architectural details of some ASes with most MPLS usage; Finally,
Sec.~\ref{ccl} concludes this paper by summarizing its main achievements.

