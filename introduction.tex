\section{Introduction}\label{intro}
% %%%%%%%%%%%%%%%%%%%%%
Internet topology refers to the study of the various types of
connectivity structures and representations between directly connected nodes on
the Internet architecture~\cite{Calvert97}. This representation aims at
obtaining models that represent  Internet with the greatest possible accuracy in order
to test new communications protocols, algorithms, QoS policies, traffic
engineering, etc.

The Internet topology can be seen at several abstraction levels i.e., IP
interface, router, subnetwork, PoP, and Autonomous System (AS) levels. All these
models have been widely studied in the past~\cite{DONNET13}. However, the
current state of the art of Internet deployments involves a great number of
technologies impacting the  Internet Topology.  And those technologies deserve a
deep study in order to include them in the current Internet models.  For instance, 
\dfn{Multiprotocol Label Switching} (MPLS)~\cite{rfc3031} has been recently the
focus of several studies~\cite{SOM11,Donnet12,Vanaubel15}.  It has been
demonstrated that MPLS is a mature technology widely deployed for (mainly) load
balancing reasons or traffic engineering purposes.  A few studies have
partially questioned its impact on Internet topology~\cite{BRICE07,Flach2012}. 
Although, the MPLS structure on the Internet architecture has not been	
studied yet.  The importance to study  the architectural details of MPLS usage 
would help to know the way in
which the Internet Service Providers (ISPs) use their networks or apply their
policies for traffic engineering as well as to better understand the today's
Internet architecture more accurately. 

This work mainly provides a study around the
structure of MPLS usage over Internet Topology. Initially, we focus on evaluate the
accuracy of MPLS tunnel detection methods. Particularly, we provide a quantification 
of the biases related with MPLS tunnels that are not revealed explicitly by \traceroute.
Secondly, we study the MPLS structure based on the way that LSRs and MPLS networks interact with non-MPLS capable routers. In order to do it, we define a new abstraction level on 
Internet graph, distinguishing  router-level and MPLS-level links. We also
identified non-MPLS capable routers and  MPLS clusters. In this way, we contribute to 
the traditional Internet Topology with new details related with MPLS usage.
We mainly use $k$-core decomposition~\cite{batagelj2002} based tools to reveal the fingerprints
closely related with MPLS presence. It has been shown
previously that $k$-core decomposition is a relevant tool to describe
Internet Topology ~\cite{Alvarez06k, Serrano06, Alvarez08k}. 
Our main findings reveal that MPLS
structure varies depending the type of MPLS tunnels that prevails for a given
AS and that MPLS deployments play an important role on Internet Backbone. Specifically,
we found that local robustness of internet topology increase due to MPLS presence.


The remainder of this paper is organized as follows: Sec.~\ref{related} provides
the state of the art and the background related to MPLS tunnels discovery. In
particular, it describes how MPLS tunnels can be revealed through active
measurements;  Sec.~\ref{dataset} explains how we collected data for this work;
Sec.~\ref{validation} presents our results related to \textit{mpls signatures}
accuracy; Sec.~\ref{cluster} presents the main contributions of this paper with
a detailed study around the behaviour of MPLS networks on the Internet Topology
and architectural details of some ASes with most MPLS usage; Finally,
Sec.~\ref{ccl} concludes this paper by summarizing its main achievements.

