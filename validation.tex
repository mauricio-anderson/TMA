\section{MPLS Signatures Validation}\label{validation}
% %%%%%%%%%%%%%%%%%%%%%%%%%%%%%%%%%%%%

In this section we describe the used methodology to validate the MPLS signatures
revealed by traceroutes. Basically, we compare the LSR position within an MPLS tunnel with the signatures values.

Implicit tunnels are based either on \textit{qTTL} or \textit{u-turn} signatures. Both of
them, directly related with MPLS position:

\begin{itemize}
  \item[i] \textit{qTTL} value refers to the IP-TTL of ICMP-\textit{echo}
  message when it enters to the MPLS  tunnel. A quoted TTL of $n$ in
  the incoming ICMP reply means that the sent probe
  expired $n$ hops later to ingress to the \textit{LSP}, i.e., a LSR reply where $\textit{qTTL}=n$ means that the LSR appears  in the $n$ position of MPLS tunnel.
  \item[ii] \textit{u-turn} value is related with the tunnel length $L$ and the $n$-position of the LSR within the tunnel (see section \ref{related.revealing})
\end{itemize}


%\begin{figure*}[!t]
%\centering
%\subfloat[Case I]{\includegraphics[width=2.5in]{box}%
%\label{fig_first_case}}
%\hfil
%\subfloat[Case II]{\includegraphics[width=2.5in]{box}%
%\label{fig_second_case}}
%\caption{Simulation results for the network.}
%\label{fig_sim}
%\end{figure*}

Our signature validation relays on the hypothesis that the MPLS position match
with the $n$-position  which an LSR is revealed by traceroute. Indeed,
we use a paris-traceroute \cite{BRICE06}  based tool in order to sure that the probes follow the
same path. In order to validate our hypotheses, first, we compare a high reliable signature such as \textit{qTTL} with  $n$-position.

To probe our assumption, the \textit{qTTL} values should to match with
$n$-position, i.e., $\textit{qTTL}=n$. The
results are showed in figure \ref{n_vs_qttl}. The figure \ref{hist_length} shows the MPLS tunnel length distribution that additionally can help us to understand the amount LSR distributed by position. We principally noticed that \textit{qTTL}
signature highly match with $n$.  The bias $\textit{qTTL}=n \pm \vartriangle $ could
occur due to the limitation in our method to reveal the first LSRs in
the tunnel if it does not implement the RFC4950. %, i.e., the first LSR revealed by traceroute within the tunnel  ($n=1$) could correspond to a LSR with $\textit{qTTL}>n$. 
A second deviation could
occur due to possible load balancer presence in the follow path. In this case,
the traceroute probes follow paths with different lengths before to reach the
MPLS tunnel. This issue
produce that the same \textit{qTTL} value could be revealed several times in different
$n$-positions. The figure also shows that \textit{qTTL} 
frequently takes the value of $1$. This usually  means that \tpropagate is not implemented. However, we found that around $2\%$ of LSRs do not react to \textit{qTTL} signature, even if \tpropagate option is activated, i.e., some LSRs interfaces located at $i_{n\pm 1}$ tunnel positions react properly to \textit{qTTL} signatures but the LSR interface located at $i_n$ position does not.
% We think that this issue occurs because some routers update the IP-TTL to $1$ in the expired packet instead of keeping it unchanged as usual. 
However, $n$-position are highly reliable. We found that in $58\%$ of the cases the $n$-position match with the
\textit{qTTL} value while in $36,3\%$ of cases the \textit{qTTL} signature is not
present and takes the value of $1$, and just $6,7\%$ of the cases presented
some bias around the expected value. This facts support our hypothesis: the  MPLS tunnel position highly match with the $n$-position  which an LSR is revealed by traceroute. Thereby,  we use
$n$-position as a reference value to validate the \textit{u-turn} signatures. The expected \textit{u-turn} value is related with $n$ in the way $\textit{u-turn}=2(L-(n-1))$, where $L$ is the tunnel length. Because
\textit{u-turn} is commonly present in almost all LSRs, first, we compare $n$ vs \textit{u-turn} on
LSRs  revealed either explicitly or \textit{qTTL} based.

Subsequently, we study $n$ value on LSRs where \textit{u-turn} was the only detected
signature. We use the filter $\textit{u-turn}>3$ to
avoid false isolated \textit{u-turn} signatures. The results for a given tunnel length $L$ are showed on figures
\ref{fig_uturn_a}, \ref{fig_uturn_b}. Similar results was noticed for other tunnel length values. Basically, we noticed that the obtained values are close to expected values on the LSRs explicitly revealed  and \textit{qTTL} based (figure \ref{fig_uturn_a}). However on the set of  LSRs revealed just by \textit{u-turn} signatures (figure \ref{fig_uturn_b}) the values commonly  does not match. If we accept a bias of $ \pm 2$ around the expected value, we
noticed that  on LSRs explicitly revealed  and \textit{qTTL} based, the $60\%$ of obtained \textit{u-turn} values match
with the expected values. However, on LSRs
revealed just trough \textit{u-turn} signature (therefore where it
is really useful), the obtained \textit{u-turn} value just match in $25\%$ with the expected value $2(L-(n-1))$. 
Therefore, LSRs revealed through \textit{u-turn} is highly inaccurate and overestimated. Mainly, because MPLS tunnels are not the
only behaviour that could causes \textit{u-turn} signatures. Indeed, has been showed that load balancing is common on traverse paths even between the same pairs source-destination \cite{BRICE07}. This issue is called \textit{per-flow} load balancing. Basically, packets that belongs to the same \textit{flow} are treated similarly \cite{BRICE06}. A \textit{flow} is identified by the first 32 bits of the IP \textit{payload}. In the case of ICMP messages, this fields refers to \textit{Type},
\textit{Code} and \textit{Checksum}. By definition, \textit{u-turn} signature is
based on two kinds of ICMP messages: ICMP \echoreply \textit{Code 11} and ICMP
\ttlexceeded \textit{Code 0}. Due to the different codes values
related with each ICMP message, there is no way to sure that they belong to the same \textit{flow} identifier and thereby to be sure that \textit{u-turn} value is caused just by MPLS tunnels.
