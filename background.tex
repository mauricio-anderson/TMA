\section{Related Work}\label{Related}
%%%%%%%%%%%%%%%%%%%%

\subsection{MPLS overview}\label{Related.mpls}
% %%%%%%%%%%%%%%%%%%%%%%%%%
Multiprotocol Label Switching (MPLS) \cite{REF} was originally designed
to speed up the forwarding process. In practice, this was done with one or more
32 bits labelstack entries(LSE) inserted between the frame header  and the IP
packet.

In a MPLS network, packets are forwarded using an exact match lookup of a 20-bit
label found in the LSE. An MPLS LSE also has a time-to-live (LSE-TTL) field and
a type-of-service field \cite{rfc1771}. At each MPLS hop, the label of the
incoming packet is replaced by a corresponding outgoing label found in an MPLS
switching table. A portion of a path where the forwarding decision is not
anymore based on longest prefix matching but rather on MPLS features is known as
an MPLS tunnel. A router with label switching capabilities is called Label
Switching Router (LSR). A series of LSRs connected together form a Label
Switched Path(LSR).

\subsection{Revealing MPLS tunnels}

Donnet et al.~\cite{Donnet12} provided a taxonomy for MPLS tunnels revealed by
traceroute and developed algotithms in order to detect it based on the way that
LSR react to ttl-propagate  and RFC 4950 option. Regarding this, we can sumarize
two kind of visible MPLS tunnels: explicit tunnels and implicit tunnels.

Explicit tunnels are based on RFC 4950 implementation. MPLS routers may send
ICMP time-exceeded messages when the LSE-TTL expires. The RFC 4950 is an
extension to ICMP allowing a router to embed an MPLS LSE in an ICMP
time-exceeded message. In that case, when the LSE-TTL expires on a MPLS router,
it simply quotes the MPLS label stack in the ICMP \texttt{time-exceeded}
message.

Implicit tunnels discovery are based on two signatures detection: \textit{qttl}
and\textit{u-turn} signature. \textit{qttl} signatures are relate with
\texttt{ttl-propagate} option, if this option is implemented, the first MPLS
router (the Ingress Label Edge Router - LER) of an LSP copies the IP-TTL value
to the LSE-TTL field rather than setting the LSE-TTL to an arbitrary value, then
LSRs along the LSP will reveal themselves via ICMP messages even if they do not
implement RFC 4950. Typically the IP packet expires when the TTL value decrease
until one ($qttl=1$). Thereby, when $qttl>1$, the ICMP reply message belongs to
an MPLS tunnel. \textit{U-turn} signature relies on the fact that most LSRs in an LSP
present a common behavior: when the LSE-TTL expires, the LSR first sends the
time-exceeded reply to the LH router wich then forwards the reply to the
probing source using directly an IP route if available. The delta between
these probing messages is identified as \textit{u-turn} signature. These
signature on MPLS tunnel discovery are discussing in detail on\cite{Donnet12} .

The \texttt{qttl} signature ($qttl>1$) is present just in MPLS behaviour.
Commonly there is not another way to get a $qttl>1$. However, \texttt{u-turn}
signature could be caused for another Internet issues such as police routing or
load balancing paths, that produces different lengths in the return path. 

In this paper we consider that RFC 4950 implementation and \textit{qttl} signature are
highly reliable methods to reveal MPLS tunnels and we mainly focus on to test
\texttt{u-turn} signature accuracy.

