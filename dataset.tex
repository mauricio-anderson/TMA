\section{Dataset}\label{dataset}
% %%%%%%%%%%%%%%%%
In order to get \textit{qTTL} and \textit{u-turn} signatures, we develop a tool
called \magallanes\footnote{\magallanes is freely available at \url{\ldots}}
allowing us to easily run and manage \scamper~\cite{Luckie10} based probes
through the PlanetLab (PL) infrastructure.  \magallanes starts by randomly
allocating several vantage points (VP) within the available set of PL nodes.  It
next distribues, among those VPs, a given number of probe targets.
To achieve some geographical uniformity in target selection, \magallanes uses
data provived by IP geolocation database maxmind.\footnote{See
\url{www.maxmind.com}.  Although we know that IP geolocation databases suffer
from strong accuracy limits~\cite{geolocation}, we believe it is enough for our
purpose as we do not need accurate geolocation.}  In this way, it chooses the
targets randomly and proportionality distributed according to the number of
subnets assigned by the Regional Internet Registry (RIR) to each region.
Additionally, \magallanes allows one to store the experiments results on a
centralized database and to perfom alias resolution using MIDAR~\cite{Keys13}.

We run \magallanes on October \nth{31}, 2015.  We choose 100 VPs and select
10,000 targets per VP.  Each VP manages its own set of targets, meaning that
probes targets are disjoint sets between VPs.  \scamper is configured to run
ICMP Paris traceroute~\cite{BRICE06}.  To get the \textit{u-turn} signature,
we send a \ping to each hop revealed by Paris traceroute. We sent six
ICMP \echorequest packets from the same VP.  Six ICMP \echoreply allow us to
infer with $95\%$ confidence if there is a single return path and, 
therefore, reduce measurement errors caused by a reverse path containing
load-balanced segments of different lengths~\cite{BRICE07}. 

As a result we discovered around 270,000 IP interfaces,  520,000 links, $42\%$
of which were available to run MIDAR and we found aliases successfully on $19\%$
of then. To match IP interfaces to ASes, we use the CAIDA
dataset~\cite{caida_ref} derived from Routeviews \footnote{See
\url{www.routeviews.org}} and collected the same day as the exploration.  Additionally we found that $44\%$ of traces collected
traverse at least one MPLS tunnel.  The amount of explicit tunnels is highly
superior to implicit ones. We discovered explicit tunnels on $34\%$ of
traceroutes and at least one implicit tunnel on $16\%$. Surprisingly we found
more implicit tunnels revealed through \textit{u-turn} signature ($12\%$) rather
than \textit{qTTL} signature ($4\%$). However, the \textit{qTTL} signature
matches with at least $63\%$ of the explicit tunnels. We discuss these results
in the next sections. Finally, we did not found opaque tunnels, confirming so
their rarity~\cite{VAN2013}.
